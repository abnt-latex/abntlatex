\section{Introdução}
  Para o documento de exemplo vai ser feita uma análise de um circuito de um filtro passivo de passa-baixas, composto por um resistor e um capacitor.

  Um filtro tem a capacidade de atenuar ou realçar a amplitude de um sinal para certas bandas de frequência do sinal, visando, por exemplo reduzir o ruído de sinais em determinado espectro, ou para resolver um grande problema nas transmissões de sinais em longa distância, já que ocorrem perdas do sinal em altas frequência, fazendo com
  que tenha uma incoerência do sinal no receptor e transmissão \cite{ref:all_about_audio}. Um filtro passivo é um filtro que não requer energia externa para operar, diferentemente dos filtros ativos, consistindo apenas de componentes passivos (indutores, capacitores e resistores). Sendo os principais filtros: passa-faixa, passa-altas e passa-baixas.

  Os filtros de passagem permitem passar apenas frequências abaixo, entre ou acima de
  um ponto de corte de frequência especificado, depende se o filtro é passa-baixas, passa-faixa ou passa-altas, respectivamente sendo possível remover certas frequências indesejadas, ou isolar uma banda no caso do passa-faixa.

  \subsection{Filtros de passagem}
    \subsubsection{Passa-baixas}
      O filtro passa-baixas (também conhecido com \textit{high cut}) é um tipo de circuito que faz com que a potência sonora em frequências a partir de uma determinada frequência especificada (chamada frequência de corte) seja reduzida à metade da potência original, ou seja, permitindo que apenas as frequências mais graves mantenham valores mais próximos de sua potência inicial.

    \subsubsection{Passa-altas}
      Com funcionamento bastante similar ao passa-baixas, o filtro passa-altas (também conhecido com \textit{low cut}) por sua vez tem a função de reduzir o ganho das frequências inferiores à uma determinada frequência de corte, permitindo que as frequências superiores à de corte mantenham uma potência mais parecida com a de origem.

    \subsubsection{Passa-faixa}
      O filtro passa-faixa tem como característica principal, permitir a passagem com
      ganho mais parecido ao do sinal original, frequências próximas à uma frequência definida (denominada frequência de ressonância), enquanto que frequências abaixo e acima de frequências especificadas (denominadas frequência de corte inferior e frequência de corte superior, respectivamente) sejam reduzidas em potência.