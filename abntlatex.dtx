% \iffalse meta-comment
%
% Copyright (C) 2024 abntlatex
%
% This file may be distributed and/or modified under the
% conditions of the LaTeX Project Public License, either
% version 1.3 of this license or (at your option) any later
% version. The latest version of this license is in:
%
% https://www.latex-project.org/lppl/
%
% and version 1.3c or later is part of all distributions of
% LaTeX version 2008-05-04 or later.
%
% \fi
%
% \iffalse
%
%<driver>\ProvidesFile{abntlatex.dtx}
%
%<class>\NeedsTeXFormat{LaTeX2e}
%<class>\ProvidesClass{abntlatex}
%<class>    [2024-07-17 v1.0 .dtx ABNT Latex]
%
%<*driver>
  \listfiles
  \documentclass{ltxdoc}
  % Packages
  \usepackage[brazil]{babel}
  \usepackage[T1]{fontenc}
  \usepackage[utf8]{inputenc}
  \usepackage{xcolor}
  \usepackage{indentfirst}
  \usepackage{hypdoc}
  \usepackage{framed}
  \usepackage{multibib}
  
  \newcites{example}{Exemplos de referências}
  %
  \EnableCrossrefs
  %\CodelineIndex
  \RecordChanges
  \begin{document}
    \DocInput{abntlatex.dtx}
  \end{document}
%</driver>
% \fi
%
% \iffalse
%
%<*class>
\PassOptionsToClass{a4paper, 12pt}{article}%
\DeclareOption*{\PassOptionsToClass{\CurrentOption}{article}}
\ProcessOptions\relax
\LoadClass{article}
%</class>
%
% \fi
% ^^A --- Configuration Documentation ---
% ^^A ---
% 
% \makeatletter
%   \newcommand\todo[1]{{\color{red}\textbf{TODO:}} #1}
%
%   \renewcommand{\ps@myheadings}{%
%     \renewcommand\@oddhead{\vbox{\rightmark\hfill\thepage\vskip3\p@\hrule}}%
%     \renewcommand\@evenhead{}%
%     \renewcommand\@oddfoot{}%
%     \renewcommand\@evenfoot{}%
%   }
%
%   \pagestyle{myheadings}
%
%   \markright{abntlatex}
%
%   \renewenvironment{theglossary}{%
%     \glossary@prologue%
%     \GlossaryParms \let\item\@idxitem \ignorespaces}%
%
%   \def\glossary@prologue{%
%     \section*{{Histórico de alterações}}%
%   }%
%
%   \def\generalname{Geral}%
%
% \makeatother
%
% ^^A --- end
%
% \changes{v1.0}{2024/07/17}{Versão inicial}
% 
% \title{A Classe \textbf{ABNT\LaTeX}\\Esse documento corresponde a \textbf{abntlatex.cls}.}
% \author{abntlatex}
% 
% \maketitle
% 
% \PrintChanges
%
% \tableofcontents
%
% \begin{center}
%   \par\rule{0.3\textwidth}{0.4pt}
% \end{center}
% \thispagestyle{empty}
% \newpage
%
% \section{Introdução}
%
%   Put text here.
%
% \section{Instalação}
%   O \textbf{ABNT\LaTeX} consiste em uma classe \textbf{abnt\LaTeX} e nos arquivos de configuração do \textbf{Bib\TeX}.
%
%   Descrição dos arquivos necessários:
%   \begin{verbatim}
%     abntlatex.ins
%     abntlatex.dtx
%     abntlatex.pdf     # Essa ducumentação :)
%     abnt.bst
%     NBR6023_2018.bib  # Exemplos de referências
%   \end{verbatim}
%   Antes de iniciar a instalação certifique-se que a classe \textbf{article} esteja instalada.
%
%   \subsection{abntlatex.cls}
%     A opção \textsl{class}, \textbf{abntlatex.cls}, é obtida rodando o comando \LaTeX\space no arquivo \textbf{abntlatex.ins}. 
%     \begin{verbatim}
%       latex abntlatex.ins
%     \end{verbatim}
%   \subsection{abntlatex.pdf}
%     O arquivo de documentação, \textbf{abntlatex.pdf}, é gerado seguindo os seguintes passos:
%     \begin{verbatim}
%       pdflatex   abntlatex.dtx
%       bibtex     example 
%       makeindex -s gglo.ist -o abntlatex.gls abntlatex.glo
%       pdflatex   abntlatex.dtx
%       pdflatex   abntlatex.dtx
%     \end{verbatim}
% \section{Opções para a classe \textbf{abntlatex}}
%   As opções que a classe aceita são as mesmas que a classe \textbf{article} aceita, sendo que a opção \textbf{a4paper} e \textbf{12pt} são enviadas previamente.
%
% \section{Pacotes usados pela classe}
%   \begin{macro}{geometry (package)}
%     Pacote para configurar as margens do documento conforme a NBR 14724/2011.
%    \begin{macrocode}
\RequirePackage[a4paper, 
                top    = 3.0cm, 
                bottom = 2.0cm, 
                left   = 3.0cm, 
                right  = 2.0cm]{geometry}
%    \end{macrocode}
%   \end{macro}
%
%   \begin{macro}{mathptmx (package)}
%   \begin{macro}{babel (package)}
%   \begin{macro}{fontenc (package)}
%   \begin{macro}{inputenc (package)}
%     Conjunto de pacotes para configurar a fonte, caracteres e os títulos para o português brasileiro, além de selecionar a fonte \textbf{Times New Roman}.
%    \begin{macrocode}
\RequirePackage{newtx}
\RequirePackage[brazil]{babel}
\RequirePackage[T1]{fontenc}
\RequirePackage[utf8]{inputenc}	
%    \end{macrocode}
%   \end{macro}
%   \end{macro}
%   \end{macro}
%   \end{macro}
%   \begin{macro}{indentfirst (package)}
%   \begin{macro}{microtype (package)}
%     Pacotes para melhorar a indentação do texto, adicionar um espaçamento no primeiro paragrafo da seção e para melhorar a justificação dos parágrafos, respectivamente.
%    \begin{macrocode}
\RequirePackage{indentfirst}
\RequirePackage{microtype}
%    \end{macrocode}
%   \end{macro}
%   \end{macro}
%
%   \begin{macro}{hyperref (package)}
%   \begin{macro}{backref (package)}
%     Pacotes para configurar as referências e citações do texto.
%    \begin{macrocode}
\RequirePackage{hyperref}
\RequirePackage[brazilian, hyperpageref]{backref}	
%    \end{macrocode}
%   \end{macro}
%   \end{macro}
%
% \section{Configuração do Bib\TeX}
%   Documento \textbf{abnt.bst} gerado a partir do pacote \href{https://ctan.org/pkg/custom-bib}{custom-bib} e editado usando como referência o documento \href{https://linorg.usp.br/CTAN/info/bibtex/tamethebeast/ttb_en.pdf}{ttb\_en} para a linguagem de pilha, que usa abordagem \textbf{postfix}, usada no \textbf{Bib\TeX}.

%   Normas utilizadas no arquivo \textbf{abnt.bst} foram escritas de acordo com o documento \textbf{NBR6023/2018}, com alguns exemplos, retirados do arquivo, nas referências (\textbf{NBR6023\_2018.bib}):
%   \begin{framed}
%     \makeatletter
%       \renewcommand*{\@biblabel}[1]{\footnotemark[#1]} ^^A --- TODO:
%     \makeatother
%     \nociteexample{7.1.1.exemplo1} 
%     \nociteexample{7.1.1.exemplo2} 
%     \nociteexample{7.1.1.exemplo3} 
%     \nociteexample{7.1.2.exemplo1} 
%     \nociteexample{7.1.2.exemplo2} 
%     \nociteexample{7.1.2.exemplo3}
%     \nociteexample{7.2.exemplo1} 
%     \nociteexample{7.2.exemplo2} 
%     \nociteexample{7.2.exemplo3} 
%     \nociteexample{7.2.exemplo4} 
%     \nociteexample{7.2.exemplo5} 
%     \nociteexample{7.2.exemplo6} 
%     \nociteexample{7.2.exemplo7}
%     \nociteexample{7.3.exemplo1}
%     \nociteexample{7.7.1.exemplo1} 
%     \nociteexample{7.7.1.exemplo2} 
%     \nociteexample{7.7.1.exemplo3} 
%     \nociteexample{7.7.2exemplo1} 
%     \nociteexample{7.7.2exemplo2} 
%     \nociteexample{7.7.4.exemplo1} 
%     \nociteexample{7.7.4.exemplo2} 
%     \nociteexample{7.7.4.exemplo3}
%     \nociteexample{7.7.5.exemplo1} 
%     \nociteexample{7.7.5.exemplo2} 
%     \nociteexample{7.7.5.exemplo3} 
%     \nociteexample{7.7.5.exemplo4} 
%     \nociteexample{7.7.6.exemplo1} 
%     \nociteexample{7.7.6.exemplo2} 
%     \nociteexample{7.7.6.exemplo3} 
%     \nociteexample{7.7.7.exemplo1} 
%     \nociteexample{7.7.7.exemplo2}
%     \nociteexample{7.8.1.exemplo1} 
%     \nociteexample{7.8.4.exemplo1}
%     \bibliographystyleexample{abnt}
%     \bibliographyexample{NBR6023_2018}
%   \end{framed}
%
% \section{Descrição dos comandos}
%
%   \DescribeMacro{\split}
%     Comando para auxiliar na divisão, com um caractere especificado, de uma lista de palavras.
%
%   \DescribeMacro{\pretext}
%     Configuração de espaçamento, segundo a NBR 14724/2011, e de paginação, segundo a NBR 14724/2011, para os elementos pré-textuais.
%
%   \DescribeMacro{\maketitle}
%     Capa conforme a NBR 14724/2011, é possível utilizar os seguintes comandos para customização dos elementos: |\university|\marg{text}, |\subtitle|\marg{text}, |\predate|\marg{text}, além desses comandos foram utilizados os comandos padrões da classe \textit{article} para construção da capa, |\title|\marg{text}, |\author|\marg{text}, |\date|\marg{text}.
%
%   \DescribeMacro{\coverpage}
%     Folha de rosto conforme a NBR 14724/2011, além dos comandos utilizados para customizar os elementos da capa é possível usar: |\documenttype|\marg{text}, |\preamble|\marg{text}, |\supervisor|\marg{text}, |\cosupervisor|\marg{text}. 
%
%     \todo{Adicionar 4.2.1.1.2: Verso. }
%
%   \DescribeEnv{abstract (env)}
%     Resumo na língua vernácula conforme a NBR 14724/2011 e a NBR6028/2021, é possível utilizar o comando |\keywords|\marg{text} para configurar as palavras-chave do documento.
%
%   \DescribeMacro{\fontfigure}
%     Comando para auxiliar na implementação da fonte de figuras, conforme a NBR 14724/2011, |\fontfigure|\marg{text}, adicionando o conteúdo em tamanho reduzido, e o texto "\textit{Fonte:\space}", como é possível ver no exemplo:
%
%   \DescribeMacro{\listoffigures}
%     Lista de ilustrações conforme a NBR 14724/2011.
%
%   \DescribeMacro{\listoftables}
%     Lista de tabelas conforme a NBR 14724/2011.
%
%   \DescribeMacro{\tableofcontents}
%     Sumário conforme a NBR 14724/2011 e NBR 6027/2012.
%
%   \DescribeMacro{\thebibliography}
%     Referências conforme a NBR 14724/2011 e a NBR 6023/2018.
%
%   \DescribeMacro{\maintext}
%      Configuração de espaçamento, segundo a NBR 14724/2011, e de paginação, segundo a NBR 14724/2011, para os elementos textuais.
%
%   \DescribeMacro{\cite}
%     Configuração da exibição da citação de referência no texto conforme a NBR 14724/2011 e a NBR 10520/2023, |\cite|\marg{text}.
%
%   \DescribeMacro{\section}
%     Título das seções primarias, com a numeração progressiva, conforme a NBR 14724/2011 e a NBR 6024/2012, |\section|\marg{text}.
%
%   \DescribeMacro{\subsection}
%     Título das seções secundarias, com a numeração progressiva, conforme a NBR 14724/2011 e a NBR 6024/2012, |\subsection|\marg{text}.

%   \DescribeMacro{\subsubsection}
%     Título das seções terciarias, com a numeração progressiva, conforme a NBR 14724/2011 e a NBR 6024/2012, |\subsubsection|\marg{text}.
%
%   \DescribeMacro{\backtext}
%     Configuração de espaçamento, segundo a NBR 14724/2011, de paginação, segundo a NBR 14724/2011, e de construção das referências, segundo a NBR 14724/2011 e a NBR 10520/2023, para os elementos pós-textuais.
%
%   \DescribeMacro{\biblabel}
%   Configuração da exibição do número da referência, segundo a NBR 14724/2011 e a NBR 10520/2023.
%
% \section{Implementação dos comandos}
%   \subsection{Comandos auxiliares}
%     \begin{macro}{\splittext}
%    \begin{macrocode}
\newcommand{\splittext}[2][,]{%
\begingroup%
    \def\start{}%
    \def\itemseparate{#1}%
    \expandafter\splitaux#2\end%
\endgroup
}
\def\splitaux#1\end{%
    \@split \start#1,\end,%
}
\def\@split #1,{%
    \ifx\end#1%
        \unskip\char46% char 46 = .
    \else%
        \ifx\start#1%
        \else%
            \unskip\itemseparate\space\ignorespaces#1%
        \fi%
        \expandafter\@split%
    \fi%
}
%    \end{macrocode}
%     \end{macro}
%
%   \subsection{Elementos pré-textuais}
%     \begin{macro}{\pretext}
%       Configurações pré-texto.
%    \begin{macrocode}
\newcommand\pretext{%
  \frenchspacing%
  \pagestyle{empty}%
  %\linespread{1.25}% TODO: necessario colocar \selectfont ou usar antes do \begin{document}
}
%    \end{macrocode}
%     \end{macro}
%     \begin{macro}{\university}
%     \begin{macro}{\subtitle}
%     \begin{macro}{\predate}
%     \begin{macro}{\documenttype}
%     \begin{macro}{\preamble}
%     \begin{macro}{\supervisor}
%     \begin{macro}{\cosupervisor}
%     \begin{macro}{\keywords}
%       Comandos para implementar a capa e a folha de rosto, todos os comandos são do tipo |\command|\marg{text}.
%    \begin{macrocode}
\newcommand*{\university}[1]{\gdef\@university{#1}}%
%    \end{macrocode}
%    \begin{macrocode}
\newcommand*{\subtitle}[1]{\gdef\@subtitle{#1}}%
%    \end{macrocode}
%    \begin{macrocode}
\newcommand*{\predate}[1]{\gdef\@predate{#1}}%
%    \end{macrocode}
%    \begin{macrocode}
\newcommand*{\documenttype}[1]{\gdef\@documenttype{#1}}%
%    \end{macrocode}
%    \begin{macrocode}
\newcommand*{\preamble}[1]{\gdef\@preamble{#1}}%
%    \end{macrocode}
%    \begin{macrocode}
\newcommand*{\supervisor}[1]{\gdef\@supervisor{#1}}%
%    \end{macrocode}
%    \begin{macrocode}
\newcommand*{\cosupervisor}[1]{\gdef\@cosupervisor{#1}}%
%    \end{macrocode}
%    \begin{macrocode}
\newcommand*{\keywords}[1]{\gdef\@keywords{#1}}%
%    \end{macrocode}
%     \end{macro}
%     \end{macro}
%     \end{macro}
%     \end{macro}
%     \end{macro}
%     \end{macro}
%     \end{macro}
%     \end{macro}
%
%     \begin{macro}{\maketitle}
%       Construção da capa, realiza a remoção da numeração da pagina, centralização horizontal do conteúdo e o título e subtítulo são centralizados verticalmente no meio da pagina independente das informações que o precede, utilizando uma \textbf{minipage}. \todo{Pode acabar ficando sobrescrito pelos autores. Para a adição dos autores foi utilizada uma tabela.}
%    \begin{macrocode}
\renewcommand{\maketitle}{%
  \begin{center}%
    \begin{minipage}[t][0.5\textheight]{\textwidth}%
      \centering
      { \ifdefined\@university%
          \large\@university\par
        \fi
      }%
      \vspace*{2cm}%
      {\large\begin{tabular}[t]{c}%
        \@author%
      \end{tabular}\par}%
        
      \vspace*{\fill}%
      {\LARGE\@title\par}%
      { \ifdefined\@subtitle%
          \large\@subtitle\par
        \fi
      }%
    \end{minipage}%

    \vspace*{\fill}%
    { \ifdefined\@predate%
        \normalsize\@predate\par
      \fi
    }%
    {\normalsize\@date}%
  \end{center}%
  \newpage%
}
%    \end{macrocode}
%     \end{macro}
%
%     \begin{macro}{\coverpage}
%       Construção da folha de rosto, realiza a remoção da numeração da pagina, centralização horizontal do conteúdo e desloca as informações do preambulo, tipo de documento e universidade para a direita sem que o texto exceda o centro da pagina.
%    \begin{macrocode}
\newcommand{\coverpage}{%
  \begin{center}%
    \begin{minipage}[t][0.33\textheight]{\textwidth}%
      \centering%
      {\normalsize\begin{tabular}[t]{c}%
        \@author%
      \end{tabular}\par}%
      \vspace*{\fill}%

      {\LARGE\@title\par}%
      { \ifdefined\@subtitle%
          \large\@subtitle\par
        \fi
      }%
    \end{minipage}
    \vspace*{\fill}%
      
    {\hspace{.45\textwidth}
    \begin{minipage}[c]{0.5\textwidth}%
      \linespread{1.0}%
      { \ifdefined\@documenttype
          \small\@documenttype\par
        \fi
      }%
      { \ifdefined\@preamble
          \small\@preamble\par
        \fi
      }%
      { \ifdefined\@university
          \small\@university\par
        \fi
      }%
    \vspace*{\fill}%
    \end{minipage}\par}%
    \vspace*{\fill}%

    { \ifdefined\@supervisor%
        \normalsize Orientador: \@supervisor \par
      \fi
    }%
    { \ifdefined\@cosupervisor%
        \normalsize Coorientador: \@cosupervisor \par
      \fi
    }%
    \vspace*{\fill}%

    { \ifdefined\@predate%
        \normalsize\@predate\par
      \fi
    }%
    {\normalsize\@date}%
  \end{center}%
  \newpage%
}
%    \end{macrocode}
%     \end{macro}
%
%     \begin{environment}{abstract (env)}
%       Construção do resumo, realiza a centralização horizontal do título, além de adicionar um campo para as palavras-chave.
%    \begin{macrocode}
\renewenvironment{abstract}{%
  \begin{center}%
    \textbf{\Large\abstractname\vspace{\z@}}%
  \end{center}%
}{%
  {\par\null\par\noindent\textbf{Palavras-chave: }}%
  \splittext[;]{\@keywords}
\newpage%
}
%    \end{macrocode}
%     \end{environment}
%
%     \begin{macro}{\listoffigures}
%     \begin{macro}{\l@figure}
%       Construção da lista de ilustrações.
%
%    \begin{macrocode}
\addto\captionsbrazil{%
  \renewcommand{\listfigurename}{{Lista de Ilustrações}}%
}%

\renewcommand\listoffigures{%
  {\centering\section*{\listfigurename}}%

  {%
    %\let\oldnumberline\numberline% TODO: Corrigir o aviso
    %\renewcommand{\numberline}[1]{\figurename~\oldnumberline{%
    %  #1\hfill\textendash\hfill}%
    %}%
    \@starttoc{lof}%
  }%
  \newpage
}
\renewcommand*\l@figure[2]{\@dottedtocline{1}{\z@}{2.3em}{#1}{#2}}%
%    \end{macrocode}
%     \end{macro}
%     \end{macro}
%
%     \begin{macro}{\listoftables}
%     \begin{macro}{\l@table}
%       Construção da lista de tabelas.
%
%    \begin{macrocode}
\renewcommand\listoftables{%
  {\centering\section*{\listtablename}}%

  {%
  %  \let\oldnumberline\numberline% TODO: Corrigir o aviso
  %  \renewcommand\numberline[1]{\tablename~{#1\hfill\textendash\hfill}}%
    \@starttoc{lot}%
  }%
  \newpage
}
\renewcommand*\l@table[2]{\@dottedtocline{1}{\z@}{2.3em}{#1}{#2}}%
%    \end{macrocode}
%     \end{macro}
%     \end{macro}
%
%     \begin{macro}{\tableofcontents}
%     \begin{macro}{\l@section}
%     \begin{macro}{\l@subsection}
%     \begin{macro}{\l@subsubsection}
%     \begin{macro}{\@tocrmarg}
%       Construção do sumário.
%       \begin{itemize}
%         \item[--] \todo{Arrumar o espaçamento de quando ocorre a quebra de linha.}
%         \item[--] \todo{Ajustar o espaçamento dos pontos.}
%         \item[--] \todo{Ver 6.1 NBR 6027/2012.}
%       \end{itemize}
%    \begin{macrocode}
\renewcommand\@tocrmarg{2.0cm}

\renewcommand\tableofcontents{%
  {\centering\section*{\contentsname}}%

  \@starttoc{toc}%
  \newpage
}
%    \end{macrocode}
%       Cria um comprimento para definir o espaçamento para a direita do conteúdo do sumário.
%    \begin{macrocode}
\newlength{\@sizecontentnumber}%
\settowidth{\@sizecontentnumber}{0.0.0\quad}%
%    \end{macrocode}
%       Configura as linhas do sumário respeitando a hierarquia da formatação das seções.
%    \begin{macrocode}
\renewcommand*\l@section[2]{\@dottedtocline{1}{\z@}%
  {\@sizecontentnumber}{\bfseries\slshape#1}{#2}%
}%
\renewcommand*\l@subsection[2]{\@dottedtocline{1}{\z@}%
  {\@sizecontentnumber}{\bfseries#1}{#2}%
}%
\renewcommand*\l@subsubsection[2]{\@dottedtocline{1}{\z@}%
  {\@sizecontentnumber}{#1}{#2}%
}%
%    \end{macrocode}
%     \end{macro}
%     \end{macro}
%     \end{macro}
%     \end{macro}
%     \end{macro}
%
%   \subsection{Elementos textuais}
%     \begin{macro}{\maintext}
%       Configuração dos elementos textuais.
%    \begin{macrocode}
\newcommand\maintext{%
  \frenchspacing%
  \pagestyle{myheadings}%
  %\linespread{1.25}% TODO: necessario colocar \selectfont ou usar antes do \begin{document}
}
%    \end{macrocode}
%     \end{macro}
%
%     \begin{macro}{\cite}
%     \begin{macro}{\@cite}
%       Configuração das citações, para o comando |\cite|\marg{text}.
%    \begin{macrocode}
\let\oldcite\cite
\renewcommand*\@cite[1]{#1}
\renewcommand*\cite[2][\@empty]{(\oldcite{#2}\ifx#1\@empty\relax\else, #1\fi)}
%    \end{macrocode}
%     \end{macro}
%     \end{macro}
%
%     \begin{macro}{\section}
%     \begin{macro}{\subsection}
%     \begin{macro}{\subsubsection}
%       Construção dos títulos das seções, respeitando a hierarquia da formatação das seções.
%    \begin{macrocode}
\renewcommand\section{%
  \@startsection{section}{1}{\z@}%
  {\z@}%
  {12pt}%
  {\normalfont\Large\bfseries\slshape}%
}
%    \end{macrocode}
%    \begin{macrocode}
\renewcommand\subsection{%
  \@startsection{subsection}{2}{\z@}%
  {-12pt}%
  {12pt}%
  {\normalfont\large\bfseries}%
}
%    \end{macrocode}
%    \begin{macrocode}
\renewcommand\subsubsection{%
  \@startsection{subsubsection}{3}{\z@}%
  {-12pt}%
  {12pt}%
  {\normalfont\large}%
}
%    \end{macrocode}
%     \end{macro}
%     \end{macro}
%     \end{macro}
%
%     \begin{macro}{\fontfigure}
%    \begin{macrocode}
\newcommand\fontfigure[1]{\caption*{\small Fonte: #1}}
%    \end{macrocode}
%     \end{macro}
%
%   \subsection{Elementos pós-textuais}
%
%     \begin{macro}{\backtext}
%       Configuração dos elementos pós-textuais.
%     \begin{macrocode}
\newcommand\backtext{%
  \frenchspacing%
  \bibliographystyle{unsrt}%
  \pagestyle{myheadings}%
  %\linespread{1.0}% TODO: necessario colocar \selectfont ou usar antes do \begin{document}
}
%    \end{macrocode}
%     \end{macro}
%
%     \begin{macro}{\biblabel}
%     \begin{macro}{\@biblabel}
%       Configuração das referência.
%    \begin{macrocode}
\renewcommand*{\@biblabel}[1]{\footnotemark[#1]}
%    \end{macrocode}
%     \end{macro}
%     \end{macro}
%
%     \begin{macro}{\thebibliography}
%     \begin{macro}{\backref}
%     \begin{macro}{\backrefalt}
%       Configuração da bibliografia.      
%    \begin{macrocode}
\def\thebibliography#1{%
  \newpage%
  \phantomsection%
  \addcontentsline{toc}{section}{\protect\numberline{}\refname}%

  {\centering\section*{\refname}}%
  \list{\@biblabel{\arabic{enumi}}}%
    {\settowidth\labelwidth{(#1)}%
      \leftmargin\labelwidth\advance\leftmargin\labelsep%
        \usecounter{enumi}%
    }%
  \def\newblock{\hskip .11em plus .33em minus .07em}%
  \sloppy\clubpenalty4000\widowpenalty4000\sfcode`\.=1000\relax%
}
%    \end{macrocode}
%       Texto padrão antes do número das páginas
%    \begin{macrocode}
\renewcommand{\backref}{}
%    \end{macrocode}
%       Define os textos no final da citação
%    \begin{macrocode}
\renewcommand*{\backrefalt}[4]{
  \ifcase #1
    Nenhuma citação no texto.
  \or
    Citado na página #2.
  \else
    Citado #1 vezes nas páginas #2.
  \fi
}
%    \end{macrocode}
%     \end{macro}
%     \end{macro}
%     \end{macro}
%
% \Finale
\endinput % End of File.